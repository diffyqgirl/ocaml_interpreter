\documentclass{article}
\title{CS51 Final Project}
\author{Laura Ungar}
\begin{document}
\maketitle

For my final project extension, I decided to implement eval\_l. It seemed like a natural choice, since OCaml itself is lexically scoped. This would have been much easier to do if I had not chosen to use a helper function with return type expression when writing my eval\_d method. This choice made sense for eval\_d, since eval\_d never deals with Closures. Repeatedly wrapping expressions in Env.Val then pattern matching to get rid of the Env.Val would have made no sense in eval\_d, and so I did not retroactively change my eval\_d method. However, for eval\_l, a helper function returning an expression would not work, since one has to return a closure for the Function case then use that closure in the function application case. This design difference is why I did not attempt to merge the two methods as was suggested in the writeup.

I started by copy-and-pasting my code from eval\_d, then modifying it to not require a helper function. The main cases that needed to be changed to implement lexical scoping were 
\begin{verbatim}
| Fun (v, exp1) -> exp
| App (exp1, exp2) -> (match eval_d_h exp1 env with
    | Fun (v, f) ->  eval_d_h f 
      (Env.extend env v (ref (Env.Val (eval_d_h exp2 env))))
    | _ -> raise (EvalError ("invalid function application")))
\end{verbatim}
from eval\_d becoming
\begin{verbatim}
| Fun (v, exp1) -> Env.Closure (exp, env)
| App (exp1, exp2) -> (match eval_l exp1 env with
    | Env.Closure (Fun (v, f), env2) -> eval_l f 
      (Env.extend env2 v (ref (eval_l exp2 env)))
    | _ -> raise (EvalError ("invalid function application")))
\end{verbatim}
The eval\_l version extends env2 (the environment passed up from the function) rather than env (the current environemnt) when evaluating function application. This is the critical difference that allows the program to "remember" the scope of the function to make miniml lexically scoped rather than dynamically scoped.

For reasons unknown to me and several TFs, who said my code for let rec in eval\_d looked fine for both functions, I also had to change my let rec case slightly. The version copied from eval\_d, which was transcribed into OCaml from the lecture slides and works perfectly for eval\_d, throws an EvalError in eval\_l because at some point it tries to directly evaluate Unassigned. The code in eval\_l is more methodical about setting up the recursive pointers and does not throw this error.

I use the test function test\_scope () to demonstrate the differences between eval\_l and eval\_d. test\_scope () runs on two cases involving variable shadowing where eval\_l and eval\_d should return different results.

\end{document}